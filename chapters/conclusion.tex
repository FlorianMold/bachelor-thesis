%%%%%%%%%%%%%%%%%%%%%%%%%%%%%%%%%%%%%%%%%%%%%%%%%%%%%%%%%%%%%%%%%%%%%%%%
\chapter{Zusammenfassung und Ausblick}
\label{sec:conclusion}
%%%%%%%%%%%%%%%%%%%%%%%%%%%%%%%%%%%%%%%%%%%%%%%%%%%%%%%%%%%%%%%%%%%%%%%%
Dieses Kapitel fasst die wichtigsten Ideen und Ergebnisse dieser Arbeit zusammen und gibt einen Ausblick über zukünftige darauf aufbauende Arbeiten.

\section{Zusammenfassung}
Das Ziel dieser Arbeit war das Entwerfen und Umsetzen eines prototypischen Serious Games für Schlaganfallpatient/innen. Es werden alle notwendigen Schritte für die Erstellung von \textit{Plan your Day} beschrieben. Die Arbeit teilt sich in fünf Hauptkapitel.

Kapitel \ref{sec:introduction} hat einen kurzen Überblick über die Motivation hinter dieser Arbeit gegeben und eine Problemstellung formuliert. Am Ende des Abschnitts wurde ein kurzer Überblick über die erwarteten Ergebnisse gegeben.

In Kapitel \ref{sec:theoretic-part} wurden alle relevanten Informationen zum Thema Schlaganfall und Serious Gaming beschrieben. Dabei wurde beschrieben, dass der Schlaganfall eine Krankheit ist, die jährlich rund 15 Millionen Menschen betrifft. Der Schlaganfall ist nach Herz- und Krebserkrankungen die dritt häufigste Todesursache. Ein Schlaganfall wird durch das plötzliche Platzen einer Gehirnarterie, welche eine Durchblutungsstörung im Gehirn hervorruft, ausgelöst. Es werden vier unterschiedliche Arten von Schlaganfällen (Ischämischer Insult, Hämorrhagischer Schlaganfall, Sinusvenenthrombose, \enquote{Schlagerl}) unterschieden. Zu den Indizien für einen Schlaganfall zählen Bewegungungsstörungen, Empfindungsstörungen, Sprachstörungen, Sehstörungen, Koordinationsstörungen und Gangunsicherheit. Zu den Ursachen für einen Schlaganfall zählen Bluthochdruck, hohe Blutfettwerte, Rauchen, Mangel an Bewegung oder hormonelle Einflüssen. Aber auch Krankheiten wie Atherosklerose, Diabetes mellitus, Vorhofflimmern, Tumore oder rheumatische Erkrankungen können Risikofaktoren sein. Die Folgeschäden eines Schlaganfalls unterscheiden sich von Person zu Person, je nach Typ, Schweregrad, Lokalisation und Anzahl der Schlaganfälle. Zur Behandlung von Schlaganfallpatient/innen sind eine Menge Expert/innen aus den verschiedensten Bereichen wie Ergotherapie, Logopädie, Physiotherapie und Neuropsychologie notwendig.

Im nächsten Abschnitt wurden Serious Games und deren Einsatz im Gesundheitsbereich beschrieben. Dabei wurden verschiedene Definitionen für Serious Games geliefert und wie sich diese von normalen Videospielen unterscheiden. Danach wurde die Wirkung von Serious Games beschrieben. Daraufhin wurde der Zusammenhang zwischen Spielen und Lernen präsentiert. Spaß ist ebenso ein wichtiges Element in Spielen, weil der Unterhaltungsfaktor ein wichtiges Element für das effektive Lernen ist. Positives Auswirkungen von Spielen sind kognitive, motorische und kommunikative Fähigkeiten. Danach wurden Kriterien für die effektive Behandlung von Schlaganfällen mit Serious Games und die Unterhaltung der älteren Bevölkerung beschrieben. Ein wichtiges Kriterium ist, dass Patient/innen eine Verbindung zwischen der Therapie und alltäglichen Aufgaben sehen.

In Kapitel \ref{sec:state-of-the-art} wurden State-of-the-Art Lösungen von Serious Games im Rehabiliationsbereich beschrieben. In diesem Bereich existieren viele Beispiele wie die in diesem Abschnitt erwähnten: \nameref{sec:rehab@home}, \nameref{sec:gardener}, \nameref{sec:agar}, \nameref{sec:reha-labyrinth}, \nameref{sec:robi-game}. Dabei wurde die entwickelte Anwendung mit den anderen Lösungen verglichen. Der Vergleich fand aufgrund von Kriterien wie Anwendungsgebiet, Interaktionstechnologie, Spieleranzahl, Feedback, Portierbarkeit Kompetitivität, Datensammlung, Anpassbarkeit und mögliche Use-Cases statt. Während die anderen Serious Games auf Eingabe mittels extra Hardware setzen, verwendet \enquote{Plan your Day} einzig und allein ein Smartphone oder Tablet.

In Kapitel \ref{sec:results} wurde der Entwicklungsprozess für \textit{Plan your Day} beschrieben. Die Spielidee wurde nach einem Besuch und einer anschließenden Führung im LK Allentsteig ausgearbeitet. Beim Besuch im Therapiezentrum wurden die Therapeut/innen nach ihren Wünschen im Bereich des Serious Gaming befragt. Dabei wurde bemängelt, dass im exekutiven Bereich noch nicht viele spielerische Ansätze existieren. Das erste Konzept wurde dem Betreuer der Arbeit vorgelegt, der einige Verbesserungsvorschläge einbrachte und die grundsätzliche Idee befürwortete. Mockups wurden verwendet um die Durchführbarkeit der Features zu testen. In diesem Kapitel wurde der fertige Prototyp beschrieben. Der Prototyp beinhaltet vier Spielabschnitte. Im ersten Abschnitt muss der/die Spieler/in Rezepte den richtigen Tageszeiten zuordnen. Im zweiten Abschnitt muss der/die Patient/in sich die Zutaten eines Rezepts einprägen, um diese später wiedergeben zu können. Im dritten Abschnitt muss der/die Patient/in eine Einkaufsliste aus den eingeprägten Zutaten zusammenstellen und abschließend müssen diese Zutaten eingekauft werden. Am Ende wurde der fertige Prototyp mit medizinischem Personal anhand eines Online Fragebogens evaluiert. Problematisch war der Fakt, dass aufgrund einiger Umstände keine direkten Tests mit Patient/innen durchgeführt werden konnten, so wie es ursprünglich geplant war. Vier Teilnehmer/innen waren bei der Evaluierung beteiligt. Der Fragebogen bestand insgesamt aus 36 Fragen. Ein Großteil der Fragen waren offen mit einigen wenigen Multiple-Choice Fragen. Für einen aussagekräftigeren Vergleich wurden den Therapeut/innen mehrere Prototypen vorgeführt. Die Therapeut/innen wurden gebeten die Videos der verschiedenen Serious Games anzusehen und anschließend die Fragen zu beantworten. Durch direkten Kontakt hätten einige Unklarheiten des Fragebogens ausgeräumt werden können und noch mehr subjektives Feedback gesammelt werden können. Trotzdessen gaben die Teilnehmer/innen \textit{Plan your Day} großteils positives Feedback, mit einigen kleinen Verbesserungsvorschlägen. Einige Therapeut/innen haben angemerkt, dass sie nicht nachvollziehen können, warum Burger kein Mittagessen ist. Diese Problematik ist allerdings dem Spielprinzip geschuldet. Keine Mahlzeit lässt sich eindeutig einem Mittagessen oder einem Abendessen zuordnen. Die Möglichkeit, dass eine Mahlzeit zwei Tageszeiten zugeordnet werden kann, ist keine Option. Dadurch würde das Spiel zu einfach werden, da jede Mahlzeit jeder Tageszeit zugeordnet werden kann.

\section{Ausblick}
Der entwickelte Prototyp birgt großes Potential. Jedoch können noch einige weitere Features umgesetzt werden, welche die Anwendung verbessern. Offen ist eine Implementierung eines wählbaren Schwierigkeitsgrades. Eine Schwierigkeitsstufe soll zu Beginn gewählt werden können. Eventuell kann auch eine Stufe aufgrund der Leistungen des/der Patient/in automatisch bestimmt werden. Durch die Auswahl eines Schwierigkeitsgrades ändern sich die Rezepte oder es gibt eine Zeitbeschränkung pro Spielabschnitt. Des Weiteren könnte ein Punktesystem in die Applikation integriert werden. Pro Spiel kann eine gewisse Anzahl an Punkten erreicht werden, die in einen Highscore einfließen. Dadurch soll es möglich sein, dass sich Patient/innen untereinander messen können. Damit soll die Motivation der Spieler/innen gesteigert werden. Allerdings sind nicht alle Schlaganfallpatient/innen auf gleichen motorischen Level. Daher ist der Vergleich unter Patient/innen nicht einfach. 
Um das Problem zu umgehen, können die Patient/innen in verschiedene Leistungsgruppen eingeteilt werden, wodurch der Highscore aussagekräftiger wäre. Eine weitere Verbesserung wäre die Einteilung der Fehler in Klassen. Aktuell gibt es keinerlei Unterscheidung bei der Schwere der Fehler. Ein weiteres Feature wäre die Anlage von weiteren Rezepten durch die Therapeut/innen mit zugehörigen Zutaten. In der ursprünglichen Planung des ersten Spielabschnitts von \textit{Plan your Day} sollten nicht nur die Mahlzeiten, sondern auch andere Tagesaktivitäten geplant werden können. Dazu werden zusätzliche Wörter zusätzlich zu den Rezepten angezeigt. Einige Wörter passen in einen Tagesablauf, während andere Wörter keinen Sinn in diesem Kontext ergeben. Dies fördert die Konzentration des/der Patient/innen.

Die Problemstellung dieser Arbeit beinhaltete nur die Konzeption und prototypische Entwicklung eines Serious Games für die Schlaganfallrehabiliation. Eine Verbesserung der Therapie für Schlaganfallpatient/innen mit Serious Games ist noch zu zeigen.

Viele Indizien deuten darauf hin, dass Serious Games Schlaganfallpatient/innen bei der Rehabilitation unterstützen. Dies deckt sich auch mit den Angaben der befragten Therapeut/innen. Die Mehrheit würde die entwickelte Applikation in der Therapie einsetzen. Die Therapeut/innen beanstanden einen Mangel an Spielen, die exekutive Funktionen trainieren. In dieser Thematik existiert aktuell sehr wenig Literatur. Mögliche weitere Arbeiten sollten weitere Forschungsergebnisse in diesem Themengebiet sammeln. Zusammenfassend wurde das Fundament für den Einsatz von exekutiven Serious Games für die Behandlung von Schlaganfallpatient/innen gelegt. Erkenntnisse dieser Arbeit können für die Entwicklung eines Produkts oder für weitere Forschung verwendet werden.