%%%%%%%%%%%%%%%%%%%%%%%%%%%%%%%%%%%%%%%%%%%%%%%%%%%%%%%%%%%%%%%%%%%%%%%
%%%		HINWEIS: In deutschen Arbeiten muss zuerst die deutsche
%%%		Kurzfassung und anschließend die englische Kurzfassung
%%%		angegeben werden.
%%%%%%%%%%%%%%%%%%%%%%%%%%%%%%%%%%%%%%%%%%%%%%%%%%%%%%%%%%%%%%%%%%%%%%%

\cleardoublepage
\phantomsection\pdfbookmark[1]{Abstract}{sec:Abstract}

%%% Deutsche Kurzfassung %%%



\begin{otherlanguage}{ngerman}

\chapter*{Kurzfassung}
Der Schlaganfall stellt ein Gesundheitsproblem dar, den viele Menschen pro Jahr erleiden. Die nachfolgenden geistigen und körperlichen Einschränkungen für den/die Betroffene/n variieren stark. Je früher der Schlaganfall erkannt und behandelt wird, desto weniger Folgeschäden entstehen. Die Therapie beginnt mit einer stationären oder ambulanten Behandlung. Danach wird der/die Patient/in an ein Therapiezentrum überwiesen, wo ein Behandlungsplan erstellt wird. Wenn die Betroffenen wieder zu Hause sind, werden die Übungen aus mangelnder Motivation häufig vernachlässigt.

Die Möglichkeit eine Rehabilitation von überall durchzuführen zu können, macht mobile Geräte zu einer interessanten Plattform für medizinische Therapien. Am Anfang stand eine intensive State-of-the-Art Recherche im Bereich der Serious Games, um unterschiedliche Lösungen zu vergleichen und zu analysieren. Um Ideen für ein mögliches therapeutisches Spiel zu gewinnen, wurde das Landesklinikum Allentsteig kontaktiert. Einige Therapeut/innen erklärten sich bereit an der Entwicklung der Applikation teilzunehmen. Danach wurde eine App für Android, die auf den Technologien Node.js, Angular und MySQL basiert, entwickelt. In enger Zusammenarbeit wurde der Prototyp mit den Expert/innen evaluiert. Das Produkt wurde mit einem starken Fokus auf Usability entwickelt. Dabei sind auch die Einschränkungen der Patient/innen miteinbezogen worden. Die Messung der Fortschritte für die Therapeut/innen und Patient/innen ist ebenfalls ein wichtiger Bestandteil der Applikation. Benutzer/innen sollen anhand von eindeutigem Feedback erkennen, ob sie sich gesteigert haben.
  
Die Erkenntnisse, die in der Arbeit gewonnen wurden, legen einen guten Grundstein für zukünftige Entwicklungen im Bereich der mobilen Serious Games zur Schlaganfallrehabilitation. Insbesondere wichtig sind die gewonnenen Daten aus der Evaluierung mit dem Therapiepersonal.

\bigskip

\section*{Schl\"usselw\"orter}
Serious Game, Schlaganfall, Prototyp, Usability, Tablet, Smartphone, Android
\end{otherlanguage}

%%% Englische Kurzfassung %%%

\begin{otherlanguage}{english}

\chapter*{Abstract}
Stroke is a health problem that many people suffer every year. The following mental and physical limitations for the person(s) affected vary greatly. The earlier the stroke is detected and treated, the less consequential damage is caused. Therapy begins with inpatient or outpatient treatment. The patient is then referred to a therapy centre where a treatment plan is drawn up. When the patient is back home, the exercises are often neglected due to lack of motivation.

The possibility to carry out rehabilitation from anywhere makes mobile devices an interesting platform for medical therapies. In the beginning there was an intensive state-of-the-art research in the field of serious games to compare and analyse different solutions. In order to gain ideas for a possible therapeutic game, the allentsteig regional hospital was contacted. Some therapists agreed to participate in the development of the application. Afterwards an app for Android based on the technologies Node.js, Angular and MySQL was developed. The prototype was evaluated in close cooperation with the experts. The product was developed with a strong focus on usability. The limitations of the patients were also taken into account. Measuring progress for therapists and patients is also an important part of the application. Users should be able to see whether they have improved by means of clear feedback.
  
The insights gained in the work lay a good foundation for future developments in the field of mobile serious games for stroke rehabilitation. Especially important are the data gained from the evaluation with therapy personnel.

\bigskip

\section*{Keywords}
Serious Game, Schlaganfall, Prototyp, Usability, Tablet, Smartphone, Android

\end{otherlanguage}
