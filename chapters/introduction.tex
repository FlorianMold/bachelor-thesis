\chapter{Einleitung}\label{sec:introduction}
Dieses Kapitel beginnt mit einer Einführung in die Problematik der Krankheit Schlaganfall und beschreibt anschließend die Motivation, welche zur Verfassung dieser wissenschaftlichen Arbeit geführt hat. Danach wird die Zielsetzung und die Methodik beschrieben. Am Ende wird ein grober Überblick über den weiteren Aufbau der Arbeit gegeben.

\section{Problemstellung}
Durchschnittlich erleidet alle sechs Minuten ein Mensch aus Österreich einen Schlaganfall. Rund 50\% der Betroffenen müssen mit mehr oder weniger schweren Einschränkungen leben. In der Liste der häufigsten Todesursachen findet sich der Schlaganfall auf Platz 3. 
Ein Schlaganfall wird durch eine Verstopfung oder das plötzliche Platzen einer Gehirnarterie, welche eine Durchblutungsstörung im Gehirn hervorruft, ausgelöst. Betroffene Hirnareale werden nicht mehr ausreichend mit Sauerstoff versorgt und sterben deswegen ab. Die Symptomatik ist abhängig von der Lokalisation des Schlaganfalls. Zu den häufigsten Auswirkungen von Schlaganfällen zählen Sehstörungen, Halbseitenlähmungen, Gleichgewichts- und Koordinationsstörungen. \cite{haring:2014:insult} \\
Als Therapiemöglichkeit können bei Schlaganfällen sogenannte Serious Games eingesetzt werden. Serious Games sind Computerspiele, die primär für das Erlangen von Wissen, körperlichen Fertigkeiten und sozialen Fähigkeiten eingesetzt werden. Solche Lernspiele kommen mittlerweile vermehrt im Bereich der Rehabilitation zum Einsatz. Exemplarisch können sie zur Verbesserung der Beweglichkeit oder zur Konzentrations- und Aufmerksamkeitsschulung angewandt werden. \cite{krueger-heike:2013:serious-game} \\

\section{Motivation}
Das Ziel der Arbeit ist, dass der Rehabilitationsprozess für Therapeut/innen und Patient/innen unterstützt wird. Serious Games ermöglichen den Einbezug der Therapie in den Alltag. Die App soll dabei als Begleittherapie fungieren, mit dem Ziel, alltägliche Fertigkeiten neu anzutrainieren. Durch die spielerische Herangehensweise an die Therapie haben die Patient/innen mehr Spaß, wodurch sich die Kooperationsbereitschaft erhöht und der Therapieerfolg signifikant steigt. Im Hintergrund speichert die App Daten zu den absolvierten Spielen, der zu behandelten Personen, die nachträglich vom Therapierenden ausgewertet werden können. Im Verlauf kann der Therapieerfolg anhand von Zahlen gemessen werden und dort gezielt angesetzt werden, wo Patient/innen Probleme haben.
Anwendungen sind großteils für speziell entwickelte Hardware erhältlich. In den letzten Jahren haben sich Bewegungssensoren im Videospielbereich etabliert. Diese frei erhältlichen und erschwinglichen Sensoren haben zu vielen kommerziellen Fitnessanwendungen geführt. Daraus folgend wurden diese Technologien auch in wissenschaftlichen Arbeiten verwendet, wie beispielsweise in \cite{baranyi:reha_labyrinth:2013}\cite{rehab@home:2016}\cite{funabashi:agar:2017}\cite{stroke_patients_receive_home_rehabilitation:2016}.
Eine spielerische Herangehensweise an die Therapie zeigt einen immensen Schub an Motivation der Patient/innen. Und die Motivation hat großen Einfluss auf die \Gls{adherence}. \cite{burke:2009:optimising}

Ein großer Grund für die Entwicklung einer mobilen Applikation ist der rasante Anstieg der Smartphone Nutzer/innen. 77\% der Österreicher/innen besaßen im Jahre 2019 bereits ein Smartphone. \cite{smartphone_user:2019:austria} \\
Mobile Geräte besitzen viele nützliche Sensoren, die für eine Therapie perfekt geeignet sind. Mit einem Gyrosensor können beispielsweise Drehbewegungen aufgezeichnet werden. Daraus folgend ist es am besten die Hardware zu nutzen, die beinahe jede/r besitzt und welche erschwinglich ist. Die entwickelte Anwendung ist leistungstechnisch nicht aufwendig und kann auf allen Geräten der letzten Jahre problemlos gespielt werden.

\section{Zielsetzung}
Ein Ziel ist, die Anforderungen der Therapeut/innen in einem Prototypen umzusetzen. Die in einer Therapie verwendeten Methoden sollen in einem Serious Game angewandt werden und durch behandelndes Personal verifiziert werden. Ein Hauptpunkt der Arbeit liegt zunächst auf einer Analyse der Anforderungen, danach auf der Findung einer Idee. Das Feedback der Therapeut/innen soll zeigen, ob sich das Konzept erfolgreich in ein Spiel umsetzen lässt, welches für eine Therapie verwendet werden kann.
Am Anfang der Arbeit steht eine ausgiebige Literaturrecherche, um sich mit dem Thema Schlaganfall, dessen Ursachen, Folgen und der Behandlung auseinanderzusetzen. Dadurch konnten Erkenntnisse gewonnen werden, die später in die Entwicklung des Spiels eingeflossen sind. Darüber hinaus fand eine Auseinandersetzung mit dem Thema \textit{Serious Gaming} statt.
Damit die Anwendung einen wirklichen Mehrwert für Patient/innen bietet, war die Meinung von Expert/innen notwendig. Dazu konnte das LK Allentsteig gewonnen werden, wo sich ein Großteil des Behandlungspersonals bereit erklärte, mitzuwirken.
Durch ein Gespräch mit den Therapeut/innen konnte zunächst ein grobes Konzept für das Spiel erarbeitet werden. Das war notwendig, um theoretische Konzepte aus der Literatur mit praktischen Erkenntnissen zu verbinden. Das gewonnene Konzept wurde mithilfe der selben Therapeut/innen weiter verfeinert. Das Personal des LK Allentsteig diente als Interviewpartner.
Die Ergebnisse dieser Arbeit sind der fertige Prototyp und die gewonnen Erkenntnisse aus der Evaluierung mit den Therapeut/innen. Die Resultate werden im aktuellen Stand der Wissenschaft eingeordnet und ein Ausblick über mögliche Weiterführungen dieses Themas, die auf dieser Arbeit fußen, die auf dieser Arbeit.

\section{Aufbau der Arbeit}
Der Aufbau der Arbeit ist in dieses Einleitungskapitel, sowie zwei theoretische und drei praktische Teile aufgeteilt. Am Ende folgt noch eine Zusammenfassung und ein Ausblick.

In Kapitel 2 werden die theoretischen Grundlagen dieser Arbeit behandelt. Dabei wird auf die Krankheit Schlaganfall genauer eingegangen. Dazu zählen die Ursachen, die Folgen und der anschließende Rehabilitationsprozess. Zusätzlich wird in jenem Kapitel noch auf Serious Gaming im Allgemeinen eingegangen. Darin enthalten ist der Spaßfaktor, die Wirkung und wie Serious Games in der Rehabilitation eingesetzt werden.

Eine Evaluierung von State of the Art Lösungen wird in Kapitel 3 behandelt. Die existierende Literatur wird kurz beschrieben und abschließend anhand von einigen vordefinierten Kriterien mit der umgesetzten Lösung verglichen. 

Im vierten Kapitel der Arbeit werden die Ergebnisse der Arbeit beschrieben. Dabei wird im Detail auf die einzelnen Iterationen eingegangen, die die Applikation durchlaufen hat. Danach wird der Begriff Usability definiert und wie die Anwendung die Kriterien der Gebrauchstauglichkeit erfüllt. Am Ende wird auf die verschiedenen Planungswerkzeuge, die zur Realisierung verwendet wurden, eingegangen. Zudem wird auf der Technologiestack der Anwendung behandelt und erläutert, warum genau diese Technologien eingesetzt worden sind.

In Kapitel 6 werden die Ergebnisse präsentiert und wissenschaftlich diskutiert. Dabei wird herausgearbeitet, wie die Applikation in einen Therapieplan eingearbeitet werden kann.

Den Abschluss bildet das siebte Kapitel, welches die Arbeit nochmal zusammenfasst. Am Ende gibt es einen Ausblick über potentielle Features, sowie weitere mögliche Forschungsansätze in diesem Themengebiet.