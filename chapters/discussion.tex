\chapter{Diskussion}\label{sec:discussion}

%Dieses Kapitel beinhaltet eine Analyse der Ergebnisse eines Online-Fragebogens. Die Therapeut/innen wurden gebeten die Videos der verschiedenen Serious Games anzusehen und anschließend die Fragen zu beantworten. Vier Teilnehmer/innen waren bei der Evaluierung beteiligt. Der Fragebogen bestand insgesamt aus 36 Fragen. Ein Großteil der Fragen waren offen mit einigen wenigen Multiple-Choice Fragen. In der Beurteilung der entwickelten Applikation erhielt Plan your Day großteils positives Feedback. Problematisch ist der Fakt, dass aufgrund einiger Umstände keine direkten Tests mit Patient/innen durchgeführt werden konnten, so wie es ursprünglich geplant war. Mithilfe des Online Fragebogens konnte zwar Feedback eingeholt werden, jedoch haben nur wenige Therapeut/innen teilgenommen. Da die Befragung ausschließlich Online durchgeführt wurde, konnte nicht direkt mit den Therapeut/innen gesprochen werden. Dadurch hätten einige Unklarheiten des Fragebogens ausgeräumt werden können und noch mehr subjektives Feedback gesammelt werden können. Trotz der geringen Teilnehmer/innenzahl kann ein Trend abgeleitet werden. Allerdings hat die Evaluierung gezeigt, dass das Spielkonzept noch einige Probleme hat. Einige Therapeut/innen haben angemerkt, dass sie nicht nachvollziehen können, warum Burger kein Mittagessen ist. Diese Problematik ist allerdings dem Spielprinzip geschuldet. Keine Mahlzeit lässt sich eindeutig einem Mittagessen oder einem Abendessen zuordnen. Die Möglichkeit, dass eine Mahlzeit zwei Tageszeiten zugeordnet werden kann, ist keine Option. Dadurch würde das Spiel zu einfach werden, da jede Mahlzeit jeder Tageszeit zugeordnet werden kann. Die Ergebnisse bieten eine gute Grundlage für weitere Forschung im Bereich der Serious Games, welche exekutive Funktionen trainieren.